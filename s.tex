% Options for packages loaded elsewhere
\PassOptionsToPackage{unicode}{hyperref}
\PassOptionsToPackage{hyphens}{url}
%
\documentclass[
]{article}
\usepackage{amsmath,amssymb}
\usepackage{iftex}
\ifPDFTeX
  \usepackage[T1]{fontenc}
  \usepackage[utf8]{inputenc}
  \usepackage{textcomp} % provide euro and other symbols
\else % if luatex or xetex
  \usepackage{unicode-math} % this also loads fontspec
  \defaultfontfeatures{Scale=MatchLowercase}
  \defaultfontfeatures[\rmfamily]{Ligatures=TeX,Scale=1}
\fi
\usepackage{lmodern}
\ifPDFTeX\else
  % xetex/luatex font selection
\fi
% Use upquote if available, for straight quotes in verbatim environments
\IfFileExists{upquote.sty}{\usepackage{upquote}}{}
\IfFileExists{microtype.sty}{% use microtype if available
  \usepackage[]{microtype}
  \UseMicrotypeSet[protrusion]{basicmath} % disable protrusion for tt fonts
}{}
\makeatletter
\@ifundefined{KOMAClassName}{% if non-KOMA class
  \IfFileExists{parskip.sty}{%
    \usepackage{parskip}
  }{% else
    \setlength{\parindent}{0pt}
    \setlength{\parskip}{6pt plus 2pt minus 1pt}}
}{% if KOMA class
  \KOMAoptions{parskip=half}}
\makeatother
\usepackage{xcolor}
\usepackage[margin=1in]{geometry}
\usepackage{graphicx}
\makeatletter
\def\maxwidth{\ifdim\Gin@nat@width>\linewidth\linewidth\else\Gin@nat@width\fi}
\def\maxheight{\ifdim\Gin@nat@height>\textheight\textheight\else\Gin@nat@height\fi}
\makeatother
% Scale images if necessary, so that they will not overflow the page
% margins by default, and it is still possible to overwrite the defaults
% using explicit options in \includegraphics[width, height, ...]{}
\setkeys{Gin}{width=\maxwidth,height=\maxheight,keepaspectratio}
% Set default figure placement to htbp
\makeatletter
\def\fps@figure{htbp}
\makeatother
\setlength{\emergencystretch}{3em} % prevent overfull lines
\providecommand{\tightlist}{%
  \setlength{\itemsep}{0pt}\setlength{\parskip}{0pt}}
\setcounter{secnumdepth}{-\maxdimen} % remove section numbering
\ifLuaTeX
  \usepackage{selnolig}  % disable illegal ligatures
\fi
\IfFileExists{bookmark.sty}{\usepackage{bookmark}}{\usepackage{hyperref}}
\IfFileExists{xurl.sty}{\usepackage{xurl}}{} % add URL line breaks if available
\urlstyle{same}
\hypersetup{
  pdftitle={2024 Travelers University Modeling Competition: CloverShield Insurance Company Modeling Problem},
  pdfauthor={Oluwafunmibi Omotayo Fasanya and Augustine Kena Adjei},
  hidelinks,
  pdfcreator={LaTeX via pandoc}}

\title{2024 Travelers University Modeling Competition: CloverShield
Insurance Company Modeling Problem}
\usepackage{etoolbox}
\makeatletter
\providecommand{\subtitle}[1]{% add subtitle to \maketitle
  \apptocmd{\@title}{\par {\large #1 \par}}{}{}
}
\makeatother
\subtitle{LightGBM Modeling Approach}
\author{Oluwafunmibi Omotayo Fasanya and Augustine Kena Adjei}
\date{December 5, 2024}

\begin{document}
\maketitle

```r

\documentclass{beamer}
\usepackage{graphicx}
\usetheme{Madrid}
\usecolortheme{dolphin}
\usepackage{ragged2e}
\usepackage{subcaption}
\usepackage{amsmath}  % For mathematical symbols and equations
\usepackage{verbatim}
\usepackage{fancyvrb}
\usepackage{color}
\usepackage{amsmath}
\usepackage{tikz}
\usetheme{default}
\usepackage{hyperref}


\title{2024 Travelers University Modeling Competition: \\ CloverShield Insurance Company Modeling Problem}
\subtitle{LightGBM Modeling Approach}
\author{Oluwafunmibi Omotayo Fasanya and Augustine Kena Adjei}
\date{December 5, 2024}

\begin{document}

\frame{\titlepage}

\section{What methods did you consider?}

\begin{frame}{Methods Considered}
\textbf{Tree-Based Models:}
\begin{itemize}
    \item \textbf{Random Forest:} Robust and captures non-linear relationships and interactions.
    \item \textbf{XGBoost:} Gradient boosting algorithm, effective for structured tabular data.
    \item \textbf{LightGBM:} Histogram-based decision tree algorithm for efficient tree construction and reduced memory consumption.
\end{itemize}

\textbf{Zero-Inflated Models:}
\begin{itemize}
    \item \textbf{Zero-Inflated Poisson (ZIP):} Models excess zeros in count data.
    \item \textbf{Zero-Inflated Negative Binomial (ZINB):} Addresses overdispersion and excess zeros.
    \item \textbf{Hurdle Model (Negative Binomial):} Separates zero and non-zero counts.
\end{itemize}
\end{frame}

# Rest of the LaTeX code follows

\end{document}

\end{document}
