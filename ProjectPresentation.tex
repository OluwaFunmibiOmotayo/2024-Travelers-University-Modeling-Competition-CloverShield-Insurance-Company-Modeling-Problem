\documentclass{beamer}\usepackage[]{graphicx}\usepackage[]{xcolor}
% maxwidth is the original width if it is less than linewidth
% otherwise use linewidth (to make sure the graphics do not exceed the margin)
\makeatletter
\def\maxwidth{ %
  \ifdim\Gin@nat@width>\linewidth
    \linewidth
  \else
    \Gin@nat@width
  \fi
}
\makeatother

\definecolor{fgcolor}{rgb}{0.345, 0.345, 0.345}
\newcommand{\hlnum}[1]{\textcolor[rgb]{0.686,0.059,0.569}{#1}}%
\newcommand{\hlsng}[1]{\textcolor[rgb]{0.192,0.494,0.8}{#1}}%
\newcommand{\hlcom}[1]{\textcolor[rgb]{0.678,0.584,0.686}{\textit{#1}}}%
\newcommand{\hlopt}[1]{\textcolor[rgb]{0,0,0}{#1}}%
\newcommand{\hldef}[1]{\textcolor[rgb]{0.345,0.345,0.345}{#1}}%
\newcommand{\hlkwa}[1]{\textcolor[rgb]{0.161,0.373,0.58}{\textbf{#1}}}%
\newcommand{\hlkwb}[1]{\textcolor[rgb]{0.69,0.353,0.396}{#1}}%
\newcommand{\hlkwc}[1]{\textcolor[rgb]{0.333,0.667,0.333}{#1}}%
\newcommand{\hlkwd}[1]{\textcolor[rgb]{0.737,0.353,0.396}{\textbf{#1}}}%
\let\hlipl\hlkwb

\usepackage{framed}
\makeatletter
\newenvironment{kframe}{%
 \def\at@end@of@kframe{}%
 \ifinner\ifhmode%
  \def\at@end@of@kframe{\end{minipage}}%
  \begin{minipage}{\columnwidth}%
 \fi\fi%
 \def\FrameCommand##1{\hskip\@totalleftmargin \hskip-\fboxsep
 \colorbox{shadecolor}{##1}\hskip-\fboxsep
     % There is no \\@totalrightmargin, so:
     \hskip-\linewidth \hskip-\@totalleftmargin \hskip\columnwidth}%
 \MakeFramed {\advance\hsize-\width
   \@totalleftmargin\z@ \linewidth\hsize
   \@setminipage}}%
 {\par\unskip\endMakeFramed%
 \at@end@of@kframe}
\makeatother

\definecolor{shadecolor}{rgb}{.97, .97, .97}
\definecolor{messagecolor}{rgb}{0, 0, 0}
\definecolor{warningcolor}{rgb}{1, 0, 1}
\definecolor{errorcolor}{rgb}{1, 0, 0}
\newenvironment{knitrout}{}{} % an empty environment to be redefined in TeX

\usepackage{alltt}
\usetheme{Madrid}

% Adjust margins globally
\setbeamersize{text margin left=5mm, text margin right=5mm}

% Include logo in title page
\title[Travelers University Modeling Competition]{2024 Travelers University Modeling Competition: \\ CloverShield Insurance Company Modeling Problem}
\subtitle{LightGBM Modeling Approach}
\author[O.F. Fasanya and A.K. Adjei]{Oluwafunmibi Omotayo Fasanya, Augustine Kena Adjei}
\date{December 5, 2024}
% Add logo
\titlegraphic{\includegraphics[width=3cm]{nebraska-n.jpg}}
\IfFileExists{upquote.sty}{\usepackage{upquote}}{}
\begin{document}

\frame{\titlepage}

% Section: Introduction
\section{Introduction}

\begin{frame}{Problem Statement}
\textbf{Objective:}
\begin{itemize}
    \item Develop a predictive model to forecast policyholder call frequency (\textit{call\_counts}) for CloverShield Insurance, based on customer and policy data.
\end{itemize}

\textbf{Goal:}
\begin{itemize}
    \item Reduce call center costs by optimizing resource allocation and improving efficiency through customer segmentation.
\end{itemize}
\end{frame}

% Section: Variable Selection
\section{Variable Selection}

\begin{frame}{Variable Selection Process}
\textbf{Handling Missing Values:}
\begin{itemize}
    \item Identified variables with high percentages of missing values and flagged them for imputation.
    \item Created missing value indicator variables to capture potential predictive information from missingness (e.g., customers without telematics data may have unique behaviors).
    \item Imputation techniques:
    \begin{itemize}
        \item \textbf{Continuous variables:} Replaced missing values with the median, which is robust to outliers.
        \item \textbf{Categorical variables:} Replaced missing values with the mode, representing the most likely category.
    \end{itemize}
\end{itemize}
\end{frame}

\begin{frame}{Variable Selection Process}
\textbf{Correlation Analysis:}
\begin{itemize}
    \item Evaluated relationships with the target variable (\textit{call\_counts}):
    \begin{itemize}
        \item \textbf{Continuous variables:} Pearson correlation coefficient.
        \item \textbf{Categorical variables:} Point-Biserial Correlation to assess the relationship between binary predictors and           \textit{call\_counts}.
    \end{itemize}
    
    \item \textit{X12m\_call\_history} showed the highest correlation (0.28) with \textit{call\_counts} among continuous variables.
    \item Others variables has a very low association (less than 0.1) with call count.
\end{itemize}

\end{frame}

\begin{frame}{Variable Selection Process}
\textbf{Categorical Encoding:}
\begin{itemize}
    \item Transformed categorical variables using one-hot encoding (dummy variables) to ensure compatibility with LightGBM.
\end{itemize}

\textbf{Multicollinearity Check:}
\begin{itemize}
    \item Examined correlation matrix for continuous variables to identify and address any strong correlations among predictors.
\end{itemize}
\end{frame}


% Section: Methods Considered
\section{Methods Considered}

\begin{frame}{Methods Considered}
\textbf{Tree-Based Models:}
\begin{itemize}
    \item \textbf{Random Forest:} Captures non-linear relationships and interactions robustly.
    \item \textbf{XGBoost:} Effective gradient boosting algorithm for structured data.
    \item \textbf{LightGBM:} Efficient histogram-based decision tree with lower memory consumption.
\end{itemize}

\textbf{Zero-Inflated Models:}
\begin{itemize}
    \item \textbf{Zero-Inflated Poisson (ZIP):} Addresses excess zeros in count data.
    \item \textbf{Zero-Inflated Negative Binomial (ZINB):} Handles overdispersion and excess zeros.
    \item \textbf{Hurdle Model:} A two-part model designed to separately handle the zero and non-zero counts
\end{itemize}
\end{frame}

% Section: Chosen Method
\section{Chosen Method}

% Slide 1: Introduction to LightGBM
\begin{frame}{Chosen Method: LightGBM (1/2)}
\textbf{Overview of LightGBM:}
\begin{itemize}
    \item LightGBM is a gradient-boosting framework that integrates decision trees (weak learners) sequentially using boosting.
    \item Unlike traditional methods, it grows trees leaf-wise:
    \begin{itemize}
        \item Selects the leaf with the maximum delta loss to grow, minimizing training loss efficiently.
    \end{itemize}
    \item Shares key advantages with XGBoost:
    \begin{itemize}
        \item Regularization, sparse optimization, bagging, parallel training, and early stopping.
        \item Flexibility to handle multiple loss functions, including Poisson for count data.
    \end{itemize}
\end{itemize}
\end{frame}

% Slide 2: Advanced Features and Results
\begin{frame}{Chosen Method: LightGBM (2/2)}
\textbf{Why LightGBM?}
\begin{itemize}
    \item Designed for high-speed computation and low memory usage.
    \item Handles large datasets and categorical variables effectively.
\end{itemize}

\textbf{Advanced Techniques:}
\begin{itemize}
    \item \textbf{Leaf-wise growth:} Focuses on growing deeper trees in areas requiring finer granularity for better accuracy.
    \item \textbf{GOSS (Gradient-based One-Side Sampling):} Accelerates training by prioritizing instances with larger gradient values.
    \item \textbf{Automatic binning:} Reduces memory consumption and improves efficiency by binning continuous features.
\end{itemize}

\end{frame}

% Section: Model Evaluation
\section{Model Evaluation}

\begin{frame}{Model Evaluation}
\textbf{Metrics:}
\begin{itemize}
    \item RMSE, MAE: Measure prediction accuracy.
    \item Poisson log-likelihood: Validates count data assumptions.
    \item R-squared: Goodness-of-fit.
\end{itemize}

\textbf{Validation Approach:}
\begin{itemize}
    \item Train-test split (80-20).
    \item Early stopping to avoid overfitting.
\end{itemize}

\textbf{Feature Importance:}
\begin{itemize}
    \item LightGBM's feature importance analysis to identify influential predictors.
\end{itemize}
\end{frame}

% Slide: Additional Variables
\begin{frame}{Potentially Useful Variables (Not in the Dataset)}
\textbf{Demographic Factors:}
\begin{itemize}
    \item Education level, marital status, employment status.
    \item Local economic conditions.
\end{itemize}

\textbf{Customer Behavioral Indicators:}
\begin{itemize}
    \item Payment history and risk profile.
    \item Duration of loyalty with the company.
    \item Frequency of customer service interactions.
\end{itemize}

\textbf{Policy and Service Features:}
\begin{itemize}
    \item Insurance add-ons or optional coverages.
    \item Payment frequency (e.g., monthly or annually).
\end{itemize}

\textbf{Geographical and Economic Indicators:}
\begin{itemize}
    \item Policyholder location.
    \item Regional economic trends.
\end{itemize}
\end{frame}


% Section: Conclusion
\section{Conclusion}

\begin{frame}{Conclusion}
\textbf{Summary:}
\begin{itemize}
    \item LightGBM was selected for its efficiency, accuracy, and suitability for count data.
    \item Key predictors such as \textit{X12m\_call\_history} and behavioral factors were instrumental in explaining \textit{call\_counts}.
    \item Model evaluation metrics confirmed the robustness of the approach.
\end{itemize}

\textbf{Future Work:}
\begin{itemize}
    \item Explore hyperparameter optimization (e.g., grid search).
    \item Incorporate external features (e.g., market trends).
\end{itemize}
\end{frame}

\end{document}
